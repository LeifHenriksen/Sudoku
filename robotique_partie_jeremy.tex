\documentclass{beamer}
\usepackage[francais]{babel}
\usepackage[T1]{fontenc}
\usepackage[utf8]{inputenc}  
\usepackage{verbatim} 
\usepackage{indentfirst}          %pour \verbatiminput{fich}
\usetheme[secheader]{Madrid}    % titre en haut
\useoutertheme{infolines}       
\setlength{\parindent}{0.5cm}

\graphicspath{{Figures/}}
\setbeamertemplate{caption}[numbered] % pour numéroter tables et figures !
%si \verb   : \begin{frame}[fragile]
%si verbatim  : \begin{frame}[containsverbatim]


\title{La robotique}
\author{Jérémy SIMIONE, Thomas BESSON , Leif HENRIKSEN}
\institute{Université de Montpellier}
\date{2019}

\begin{document}
% Premier transparent de titre
\begin{frame}
  \titlepage
\end{frame}

% 2eme transparent TDM générale
\begin{frame}
  \frametitle{Sommaire}
  {\small \tableofcontents[hideallsubsections]}
\end{frame}

\section{Microcontroleurs ou nano-ordinateurs?}
\section{Qu'est ce qu'un microcontrôleur ?}

\subsection{Definition}
\begin{frame}
\begin{block}{Definition}
Un microcontrôleur est un circuit intégré qui rassemble les éléments essentiels d'un ordinateur.
 Les microcontrôleurs se caractérisent par un plus haut degré d'intégration, une plus faible consommation électrique,
 une vitesse de fonctionnement plus faible  et un coût réduit par rapport aux microprocesseurs  utilisés dans les ordinateurs personnels. 
\end{block}
\end{frame}
\subsection{Composition d'un microcontroleur}
\begin{frame}
  Les composants d'un  microcontroleur sont  :
  \begin{itemize}
      \item Un processeur ou un microprocesseur 
      \item De la mémoire vive (RAM) 
      \item De la mémoire morte (ROM) 	
      \item Des périphériques, capables d'effectuer des tâches spécifiques. On peut mentionner  :
		\begin{itemize}
   			\item Les convertisseurs analogiques-numériques (CAN) 
 			  \item Les convertisseurs numériques-analogiques (CNA) 
 			  \item Les contrôleurs de bus de communication
		\end{itemize} 
  \end{itemize}
\end{frame}

\section {Programmation d'un microcontroleur}
\subsection{Définiton}
\begin{frame}
\begin{block}{Définition}
 La principale différence entre le microcontroleur et le microprocesseur est qu'en plus de posséder une unité de calcul, le microcontrôleur possède en interne le programme qu'il devra effectuer.
 Ce mode de fonctionnement convient particulièrement bien aux applications dites "embarquées".
 Différents objets  tels que le clavier d'ordinateur ou la souris possèdent un microcontrôleur.
\end{block}
\end{frame}
\begin{frame}
\begin{itemize}
\item À l'origine, les microcontrôleurs se programmaient en assembleur (problèmes de maintenance)
\item Désormais, on utilise de plus en plus des langages de haut niveau, notamment le langage C.
\item Avec l’augmentation de la puissance et de la quantité de mémoire de stockage (FLASH) , les programmes  peuvent désormais être écrits en C++. 
\item Il existe même des frameworks et plateformes en C++ destinés à l’embarqué, comme Qtopia, mais l'utilisation de ceux-ci restera limitée aux microcontrôleurs les plus puissants. 
\end{itemize}
\end{frame}
\begin{frame}
\begin{itemize}
\item  Pour programmer le microcontrôleur, il est alors possible d'utiliser différents langages de programmations tels que:
\begin{itemize}
  \item BASIC
  \item C
  \item C++
  \item JAVA
  \end{itemize}
\end{itemize}
\begin{itemize}
  \item Le programme réalisé dans le langage de haut niveau est compilé dans le langage assembleur conçu par le constructeur du microcontroleur. Puis ce programme ainsi compilé sera injecté du PC dans la mémoire programmable du microcontroleur.
\end{itemize}
\end{frame}

\section{Nano-ordinateurs ou ordinateurs mono-carte}

\begin{frame}
 \begin{block}{Définition}
 Un ordinateur à carte unique ou ordinateur mono-carte est un ordinateur complet construit sur un circuit imprimé,
 avec un ou plusieurs microprocesseur(s), de la mémoire, des lignes d'entrée/sortie et d'autres éléments pour en faire un ordinateur fonctionnel. 
 \end{block}
\end{frame}
\begin{frame}
\begin{itemize}
\item Si un nano-ordinateur a l’apparence d’un microcontroleur, il est utile de rappeler que ce dernier peut être qualifié de nano-ordinateur dans la mesure où :
\begin{itemize}
    \item Il permet de faire fonctionner un système d’exploitation de haut niveau, comme par exemple Windows, Androïd ou Linux .
    \item Il intègre ou il permet de connecter à minima des périphériques d’affichage (écran, télévision) et de saisie (clavier physique ou virtuel, souris ou écran tactile).
    \item Il comporte un moyen de stockage dédié (disque dur, carte SD, clé usb) et possède un port ethernet.
    \item Il fonctionne de manière autonome, sans nécessité de le raccorder à un autre ordinateur.
   \item Jusqu'a 40 fois plus rapide qu'un microcontroleur
\end{itemize}
\end{itemize}
\end{frame} 
\begin{frame}
\begin{itemize}
    \item Au niveau de la programmation le nano-ordinateur pourra etre programmé dans n'importe quel langage de haut niveau et l'IDE pourra etre choisi car il comporte son propre systeme d'exploitation a contrario du microcontroleur.
    \item L'injection du code dans le nano-ordinateur se fera par un outil de transfert du PC au nano-ordinateur.
\end{itemize}
\end{frame}
\begin{frame}
\begin{itemize}
    \item Pour conclure la différence entre un microcontrôleur et un nano-ordinateur reside dans le fait qu'un nano-ordinateur est en fait un vrai PC avec la possibilité de brancher des periphériques audio ,vidéo, saisie, contrairement au microcontroleur qui sera plus approprié dans le cas de taches industrielles repetitives.
  \item Un microcontroleur sera par exemple adapté a un robot qui trie des skittles (robot industiel) , alors qu'un robot avec assistance vocale (ex: androide) aura besoin d'une connexion internet pour fonctionner ce qui est donc plus adapté a un nano-ordinateur.
\item Pour conclure le cerveau du robot peut être crée soit avec un nano-ordinateur soit avec un microcontrôleur le choix sera effectué en fonction des besoins du robot.
\end{itemize}
\end{frame}






\end{document} 

